% Chapter Template

\chapter{Bibliography} % Main chapter title

\label{Bibliography} % Change X to a consecutive number; for referencing this chapter elsewhere, use \ref{ChapterX}

%----------------------------------------------------------------------------------------
%	BIBLIOGRAPHY
%----------------------------------------------------------------------------------------

\small
\noindent
A. Biswal, “Recurrent neural network (RNN) tutorial: Types and examples [updated]: Simplilearn,” Simplilearn.com, 11-Aug-2022. [Online]. \newline Available: https://www.simplilearn.com/tutorials/deep-learning-tutorial/rnn. [Accessed: 02-Oct-2022]. \\
\newline
A. Graves, A.-rahman Mohamed, and G. Hinton, “Speech recognition with deep recurrent neural networks,” arXiv.org, 22-Mar-2013. [Online]. \newline Available: https://arxiv.org/abs/1303.5778. [Accessed: 02-Oct-2022]. \\
\newline
A. Graves, S. Fernandez, F. Gomez, and J. Schmidhuber, “Connectionist Temporal Classification: Labelling unsegmented sequence ...,” Department of Computer Science, 2006. [Online]. \newline Available: https://www.cs.toronto.edu/~graves/icml\_2006.pdf. [Accessed: 02-Oct-2022]. \\
\newline
A. Hannun, “Sequence Modeling with CTC,” Distill, 03-Jan-2020. [Online]. \newline Available: https://distill.pub/2017/ctc/. [Accessed: 02-Oct-2022]. \\
\newline
A. Lindgren and G. Lind, “Language Classification Using Neural Networks,” Uppsala Universitet, Jun-2019. [Online]. \newline Available: https://www.diva-portal.org/smash/get/diva2:1324945/FULLTEXT01.pdf. [Accessed: 02-Oct-2022]. \\
\newline
A. Taspinar, “A guide for using The wavelet transform in machine learning,” ML Fundamentals, 21-Dec-2018. [Online]. \newline Available: https://ataspinar.com/2018/12/21/a-guide-for-using-the-wavelet-transform-in-machine-learning/. [Accessed: 02-Oct-2022]. \\
\newline
B. Ramakrishnan, “Lecture 16: Connectionist Temporal Classification, sequence prediction,” YouTube, 02-May-2021. [Online]. \newline Available: https://www.youtube.com/watch?v=RowViowx1Bg. [Accessed: 02-Oct-2022]. \\
\newline
B. Ramakrishnan, “Lecture 17: Connectionist Temporal Classification (CTC), sequence to sequence prediction,” YouTube, 02-Apr-2021. [Online]. \newline Available: https://www.youtube.com/watch?v=5Rj0J9AuGw0. [Accessed: 02-Oct-2022]. \\
\newline
C. Diatkine, “Linear Predictive coding,” Introduction - Linear Predictive Coding, 2011. [Online]. \newline Available: https://support.ircam.fr/docs/AudioSculpt/3.0/co/LPC.html. [Accessed: 02-Oct-2022]. \\
\newline
H. Bidgoli, Encyclopedia of Information Systems. Amsterdam: Acad. Press, 2003. \\
\newline
H. Scheidl, “An intuitive Explanation of Connectionist Temporal Classification,” Medium, 07-Jul-2021. [Online]. \newline Available: https://towardsdatascience.com/intuitively-understanding-connectionist-temporal-classification-3797e43a86c. [Accessed: 02-Oct-2022]. \\
\newline
IBM Cloud Education, “What are recurrent neural networks?,” IBM, 14-Sep-2020. [Online]. \newline Available: https://www.ibm.com/cloud/learn/recurrent-neural-networks. [Accessed: 02-Oct-2022]. \\
\newline
“Introduction to the discrete wavelet transform (DWT),” Machine Intelligence Lab, 15-Feb-2004. [Online]. \newline Available: https://mil.ufl.edu/nechyba/www/eel6562/course\_materials/t5.wavelets/intro\_dwt.pdf. [Accessed: 02-Oct-2022]. \\
\newline
J. O. Smith, “Spectral Audio Signal Processing,” CCRMA, 2011. [Online]. \newline Available: https://ccrma.stanford.edu/~jos/sasp/. [Accessed: 02-Oct-2022]. \\
\newline
J. S. Garofolo et al., “Timit acoustic-phonetic continuous speech corpus,” Philadelphia: Linguistic Data Consortium, 1993. [Online]. \newline Available: https://catalog.ldc.upenn.edu/LDC93S1. [Accessed: 02-Oct-2022]. \\
\newline
J. Zhang, “English speech recognition system model based on computer-aided function and neural network algorithm,” Computational Intelligence and Neuroscience, 22-Apr-2022. [Online]. \newline Available: https://www.hindawi.com/journals/cin/2022/7846877/. [Accessed: 02-Oct-2022]. \\
\newline
K. Daqrouq, A. R. Al-Qawasmi, K. Y. Al Azzawi, and T. A. Hilal, “Discrete wavelet transform \& linear prediction coding based method for speech recognition via Neural Network,” IntechOpen, 12-Sep-2011. [Online]. \newline Available: https://www.intechopen.com/chapters/19512. [Accessed: 02-Oct-2022]. \\
\newline
K. S. Rao and M. K. E, “Speech Recognition Using Articulatory and Excitation Source Features,” SpringerBriefs in Speech Technology, 2017. [Online]. \newline Available: https://link.springer.com/content/pdf/bbm:978-3-319-49220-9/1.pdf. [Accessed: 02-Oct-2022]. \\
\newline
L. L. Thomala, “Baidu: R\&D spending 2021,” Statista, 01-Apr-2022. [Online]. \newline Available: https://www.statista.com/statistics/1079978/china-baidu-research-and-development-costs/. [Accessed: 02-Oct-2022]. \\
\newline
L. Li, “Performance analysis of objective speech quality measures in Mel domain,” Science Alert, 03-Jan-2015. [Online]. \newline Available: https://scialert.net/fulltext/?doi=jse.2015.350.361. [Accessed: 02-Oct-2022]. \\
\newline
N. S. Nehe and R. S. Holambe, “DWT and LPC based feature extraction methods for Isolated word recognition - eurasip journal on audio, speech, and music processing,” SpringerOpen, 30-Jan-2012. [Online]. \newline Available: https://asmp-eurasipjournals.springeropen.com/articles/10.1186/1687-4722-2012-7. [Accessed: 02-Oct-2022]. \\
\newline
P. Koehn, Neural machine translation. Cambridge, ON: Cambridge University Press, 2020.  \\
\newline
P. V. Janse, S. B. Magre, P. K. Kurzekar, and R. R. Deshmukh, “A comparative study between MFCC and DWT feature extraction technique,” International Journal of Engineering Research \& Technology, 30-Jan-2014. [Online]. \newline Available: https://www.ijert.org/a-comparative-study-between-mfcc-and-dwt-feature-extraction-technique. [Accessed: 02-Oct-2022]. \\
\newline
R. M. Gray, “Linear predictive coding and the Internet Protocol,” The Essence of Knowledge, 2010. [Online]. Available: https://ee.stanford.edu/~gray/lpcip.pdf. [Accessed: 02-Oct-2022]. \\
\newline
R. Wang, “Automatic speech recognition 101: How ASR works,” Dialpad. [Online]. Available: https://www.dialpad.com/blog/automatic-speech-recognition/. [Accessed: 02-Oct-2022]. \\
\newline
S. A. Alim and N. K. A. Rashid, “Some commonly used speech feature extraction algorithms,” IntechOpen, 12-Dec-2018. [Online]. \newline Available: https://www.intechopen.com/chapters/63970. [Accessed: 02-Oct-2022]. \\
\newline
S. M, “Let's understand the problems with recurrent neural networks,” Analytics Vidhya, 10-Jul-2021. [Online]. \newline Available: https://www.analyticsvidhya.com/blog/2021/07/lets-understand-the-problems-with-recurrent-neural-networks/. [Accessed: 02-Oct-2022]. \\
\newline
Saranga-K-Mahanta-google and arvindpdmn, “Audio Feature Extraction,” Devopedia, 23-May-2021. [Online]. \newline Available: https://devopedia.org/audio-feature-extraction. [Accessed: 02-Oct-2022]. \\
\newline
Sonix Authors, “A brief history of speech recognition,” Sonix, 2022. [Online]. \newline Available: https://sonix.ai/history-of-speech-recognition. [Accessed: 02-Oct-2022]. \\
\newline
Summa Linguae Authors, “Speech recognition software: History, present, and future,” Summa Linguae, 18-Jun-2021. [Online]. \newline Available: https://summalinguae.com/language-technology/speech-recognition-software-history-future/. [Accessed: 02-Oct-2022]. \\
\newline
V. Lendave, “LSTM vs gru in recurrent neural network: A comparative study,” Analytics India Magazine, 28-Aug-2021. [Online]. \newline Available: https://analyticsindiamag.com/lstm-vs-gru-in-recurrent-neural-network-a-comparative-study/. [Accessed: 02-Oct-2022]. \\
\newline
Y. Wang, X. Deng, S. Pu, and Z. Huang, “Residual convolutional CTC networks for automatic speech recognition,” arXiv.org, 24-Feb-2017. [Online]. \newline Available: https://arxiv.org/abs/1702.07793. [Accessed: 02-Oct-2022].